Dans ce projet, nous avons vu une méthode de reconnaissance faciale
basée sur les eigenfaces et utilisant un classifieur bayésien. La 
méthode se distingue des approches actuelles par le faite qu'elle soit
très rapide à implémenter (le modèle est généré rapidement).

Coté taux de reconnaissance, on arrive à un taux de près de $70\%$, ce qui
est bien loin des performances des systèmes actuels basés sur l'apprentissage
profond.

Ce système de reconnaissance est assez robuste aux bruits speckle et de poisson,
il l'est beaucoup moins aux bruits gaussiens et poivre et sel. Il n'est également
pas du tout robuste à l'inversion de contraste et est très peu robuste à la translation.
Pour avoir une assez bonne robustesse à la rotation, il faut introduire des images
retournées lors de la construction du modèle (dans les images d’entraînement).